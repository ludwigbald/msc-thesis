%%%%%%%%%%%%%%%%%%%%%%%%%%%%%%%%%%%%%%%%%%%%%%%%%%%%%%%%%%%%%%%%%%%%
% Diskussion und Ausblick
%%%%%%%%%%%%%%%%%%%%%%%%%%%%%%%%%%%%%%%%%%%%%%%%%%%%%%%%%%%%%%%%%%%%

\chapter{Discussion}
  \label{Discussion}

\todo[inline]{
Of course very important! You need to discuss the informatics as well as econ part of your thesis topic.

\bigskip
Take your time for writing the discussion, besides the introduction chapter it is the most important chapter of your thesis.
Also do not subsection the discussion too heavily.

At least 5 pages,

Outlook can become an extra chapter.
}

\todo[inline]{
  Outline/structure for this chapter:
  TODO: Fill with content

  - restate research question: Does an uncertainty-aware algorithm outperform other methods on this problem?
  - summarize key findings (answer question)
    - On this problem, UA-DQN learns faster than other methods. --> It explores more efficiently.
    - Its final performance is not significantly better than others (compute p value maybe, do t-test. Does that make sense?)
    - For a simple system, a tuned rule-based policy can outperform RL methods.
    - UA-DQN takes more computational resources.
  - explain and interpret results in more detail
    - relate back to other literature
    - support conclusions with data from results
    - ...
  - Discuss limitations
    - Internal Validity: Does the experiment support the conclusion?
      - Experiment was not thorough enough
      - Absence of good performance does not mean good performance is impossible, maybe I just did it wrong
      - Data was not fully used
    - External Validity: Does the conclusion generalize to real life?
      - Model limitations: Does CityLearn represent the real life application?
        - The real life application of such a system depends on technical and regulatory details
        - Data is real life, but other data might be available.
        - A real life application could have different goals than only price and CO2, like coordination goals and communication requirements.
      - 
    - Is the question important?
      - what should such a system even be judged on in application?
  - Further Research
    - Application in a real life system
    - 
}

\todo{mention that e-greedy epsilon was not tuned, but it definitely should have been!}
- Tuning: Initialization might have been different for different algorithms

- What would a good setting for the risk-aversion parameter even be?

\subsubsection{Known Limitations of the RL environment} %TODO maybe this should be in Discussion only?
- data: Weather forecasts are not real forecasts
- energy model: Battery charging speed is symmetric
- market model: No selling the electricity
- Coordination goals: Were introduced later in the challenge, not used here.
    - Only single building.

CityLearn as a model for building-based demand response:
- limitations: modelling errors/simplifications and how much they matter
    - CityLearn was built to assess the general usefulness of RL for DR, not for my specific question.
    - Battery efficiency
    - Weather Forecasts are perfect oracles
    - Only available action is battery charging/discharging.
    - Other processes are assumed to be fixed (at least in 2022 challenge). Automatic Washing Machine starting.
    - Human behavior can not be influenced (models humans as inflexible)
    - Hourly control instead of continuous -> Makes perfect control more difficult
    - Energy can not be sold to the grid, not even to the microgrid. -> unrealistic
    - 2022 challenge: initially did not encourage cooperation.
- Strengths: For what tasks is this environment adequate?
    - Makes it possible to apply RL to DR without expensive computational overload*
    - Test Multi-Agent behavior and cooperation in the DR context, as opposed to only single-building frameworks
    - Enables comparison of different approaches as a benchmark