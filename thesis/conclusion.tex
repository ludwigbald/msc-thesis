%%%%%%%%%%%%%%%%%%%%%%%%%%%%%%%%%%%%%%%%%%%%%%%%%%%%%%%%%%%%%%%%%%%%
% Zusammenfassung
%%%%%%%%%%%%%%%%%%%%%%%%%%%%%%%%%%%%%%%%%%%%%%%%%%%%%%%%%%%%%%%%%%%%

\chapter{Conclusion and Outlook}
  \label{conclusion}
% Summary
UA-DQN is a Reinforcement Learning algorithm based on the DQN algorithm.
It learns the reward distribution and provides estimates for both epistemic and aleatoric uncertainty.
This allows it to explore more efficiently, and act according to a risk-aware strategy.
In this thesis, I apply the UA-DQN algorithm to a custom task for building energy management implemented in CityLearn.
I find that tuned UA-DQN outperforms two variants of DQN, reaching comparable performance with fewer environment interactions.
However, this added efficiency comes at a cost: UA-DQN has roughly three times as much computational cost as the simpler algorithms.
None of the tuned algorithms outperform an optimized rule-based control policy.

% Conclusions
Like other algorithms that explicitly represent uncertainty in addition to point estimates, UA-DQN is able to make use of this additional information.
Application of UA-DQN and similar algorithms should therefore be considered in problems where additional observations are expensive or bad decisions are costly.
Additionally, UA-DQN might perform particularly well on non-stationary problems, when local weather patterns or occupant behavior changes.
Future work can also generalize the method behind UA-DQN to a continuous action space.

CityLearn is a suitable framework to model a building energy management task for Reinforcement Learning, but it remains a challenge to transfer insights into real-world applications because of modeling and data decisions.
Future CityLearn challenges should play to its strengths and focus on longer-term planning rather than short-term control.
% Future work
In the context of automated Demand Response by building energy systems, there remain many exciting challenges for Uncertainty-Aware Reinforcement Learning.
One interesting line of work is multiagent learning and coordination mechanisms, as they will allow direct optimization for grid-scale goals.
Future work on uncertainty-aware Reinforcement Learning methods for building control should seek to incorporate uncertain data like weather forecasts or models of occupant behavior.