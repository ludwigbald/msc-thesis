%%%%%%%%%%%%%%%%%%%%%%%%%%%%%%%%%%%%%%%%%%%%%%%%%%%%%%%%%%%%%%%%%%%%%%%%%%%%%
%%% LaTeX-Rahmen fuer das Erstellen von Masterarbeiten
%%%%%%%%%%%%%%%%%%%%%%%%%%%%%%%%%%%%%%%%%%%%%%%%%%%%%%%%%%%%%%%%%%%%%%%%%%%%%

%%%%%%%%%%%%%%%%%%%%%%%%%%%%%%%%%%%%%%%%%%%%%%%%%%%%%%%%%%%%%%%%%%%%%%%%%%%%%
%%% allgemeine Einstellungen
%%%%%%%%%%%%%%%%%%%%%%%%%%%%%%%%%%%%%%%%%%%%%%%%%%%%%%%%%%%%%%%%%%%%%%%%%%%%%

\documentclass[oneside,12pt,a4paper]{report}
\usepackage{epsf}
\usepackage{graphics, graphicx}
\usepackage{latexsym}
\usepackage[margin=10pt,font=small,labelfont=bf]{caption}
\usepackage[utf8]{inputenc}
\usepackage[toc,page]{appendix}
\usepackage{amsmath, amssymb}
\usepackage{algorithm, algpseudocode}
\usepackage{siunitx}


\usepackage[round,comma]{natbib}\bibliographystyle{plainnat}
\usepackage[backref=section, hidelinks]{hyperref}
\usepackage{url}
\usepackage{makecell}

\usepackage{todonotes}

\newcommand{\figurewidth}{0.9\linewidth}
\newcommand{\done}[2][]{\todo[color=green!40, #1]{\textbf{Done:} #2}}


\textwidth 14cm
\textheight 22cm
\topmargin 0.0cm
\evensidemargin 1cm
\oddsidemargin 1cm
%\footskip 2cm
\parskip0.5explus0.1exminus0.1ex

% Kann von Student auch nach pers\"onlichem Geschmack ver\"andert werden.
\pagestyle{headings}

\sloppy

\begin{document}

%%%%%%%%%%%%%%%%%%%%%%%%%%%%%%%%%%%%%%%%%%%%%%%%%%%%%%%%%%%%%%%%%%%%%%%%%%%%
%%% hier steht die neue Titelseite 
%%%%%%%%%%%%%%%%%%%%%%%%%%%%%%%%%%%%%%%%%%%%%%%%%%%%%%%%%%%%%%%%%%%%%%%%%%%%
 
\begin{titlepage}
 \begin{center}
  {\LARGE Eberhard Karls Universit\"at T\"ubingen}\\
  {\large Mathematisch-Naturwissenschaftliche Fakult\"at \\
Wilhelm-Schickard-Institut f\"ur Informatik\\[4cm]}
  {\huge Master Thesis Computer Science\\[2cm]}
  {\Large\bf  Uncertainty-Aware Reinforcement Learning for Demand Response in Energy Systems\\[1.5cm]}
 {\large Ludwig Bald}\\[0.5cm]
  January 31, 2023\\[4cm]
{\small\bf Reviewers}\\[0.5cm]
  \parbox{7cm}{\begin{center}{\large Dr. Nicole Ludwig}\\
  {\footnotesize Wilhelm-Schickard-Institut für Informatik\\
	Universit\"at T\"ubingen}\end{center}}\hfill\parbox{7cm}{\begin{center}
  {\large Jun. Prof. Dr.-Ing.\\ Setareh Maghsudi}\\
  {\footnotesize Wilhelm-Schickard-Institut f\"ur Informatik\\
	Universit\"at T\"ubingen}\end{center}
 }
  \end{center}
\end{titlepage}

%%%%%%%%%%%%%%%%%%%%%%%%%%%%%%%%%%%%%%%%%%%%%%%%%%%%%%%%%%%%%%%%%%%%%%%%%%%%
%%% Titelr"uckseite: Bibliographische Angaben
%%%%%%%%%%%%%%%%%%%%%%%%%%%%%%%%%%%%%%%%%%%%%%%%%%%%%%%%%%%%%%%%%%%%%%%%%%%%

\thispagestyle{empty}
\vspace*{\fill}
\begin{minipage}{11.2cm}
\textbf{Bald, Ludwig:}\\
\emph{Uncertainty-Aware Reinforcement Learning for Demand Response in Energy Systems}\\ Master Thesis Computer Science\\
Eberhard Karls Universit\"at T\"ubingen\\
Thesis period: July 2022-January 2023
\end{minipage}
\newpage

%%%%%%%%%%%%%%%%%%%%%%%%%%%%%%%%%%%%%%%%%%%%%%%%%%%%%%%%%%%%%%%%%%%%%%%%%%%%

\pagenumbering{roman}
\setcounter{page}{1}

%%%%%%%%%%%%%%%%%%%%%%%%%%%%%%%%%%%%%%%%%%%%%%%%%%%%%%%%%%%%%%%%%%%%%%%%%%%%
%%% Seite I: Zusammenfassug, Danksagung
%%%%%%%%%%%%%%%%%%%%%%%%%%%%%%%%%%%%%%%%%%%%%%%%%%%%%%%%%%%%%%%%%%%%%%%%%%%%


\section*{Abstract}
The electrical grid is increasingly dominated by renewable energy.
Therefore, demand needs to flexibly respond to changes in the amount of available electricity.
Buildings play a key role in this regard:
They consume a large amount of energy for space heating and cooling as well as providing hot water.
Intelligently controlled, they can flexibly consume energy when supply is available.

Distributional Reinforcement Learning is a class of machine learning algorithms that can efficiently explore an RL environment and learn a control policy by explicitly modeling uncertainty.
Among those, UA-DQN is an algorithm that explicitly estimates epistemic uncertainty, caused by a lack of data, and aleatoric uncertainty, caused by the environment's inherent stochasticity.

I evaluate the performance of UA-DQN on a custom RL environment for building energy management based on the 2022 CityLearn challenge.
After tuning the hyperparameters, I evaluate the algorithm against a rule-based baseline and two variants of DQN, using $\epsilon$-greedy and Boltzmann (Softmax) action selection.
All three algorithms make use of the battery, but none outperforms the baseline.
UA-DQN requires fewer environment interactions to arrive at a superior policy than both DQN variants, but demands more computational resources.

% LTeX: language=de-DE
\newpage
\section*{Zusammenfassung}
Da die Stromproduktion zunehmend auf erneuerbare Energien umgestellt wird, muss die Nachfrage flexibel auf Veränderungen der verfügbaren Produktionskapazität reagieren.
Gebäude spielen in dieser Hinsicht eine wichtige Rolle:
Sie verbrauchen einen großen Teil der Energie für Heizung und Kühlung sowie für Warmwasser.
Intelligent gesteuert können sie flexibel dann Energie verbrauchen, wenn sie verfügbar ist.

Distributional Reinforcement Learning ist eine Klasse von Algorithmen des maschinellen Lernens, die durch explizite Modellierung von Unsicherheit eine RL-Umgebung effizient erkunden und eine Steuerungspolitik erlernen können.
UA-DQN ist ein Algorithmus, der explizit die epistemische Unsicherheit, die durch einen Mangel an Daten verursacht wird, und die aleatorische Unsicherheit, die durch die inhärente Stochastizität der Umgebung verursacht wird, schätzt.

Ich bewerte die Leistung von UA-DQN in einer benutzerdefinierten RL-Umgebung für das Energiemanagement von Gebäuden auf der Grundlage der CityLearn Challenge 2022.
Ich evaluiere den Algorithmus mit optimierten Hyperparametern im Vergleich zu einer regelbasierten Baseline und zwei Varianten von DQN, die $\epsilon$-greedy und Boltzmann (Softmax) zur Aktionsauswahl verwenden.
Alle drei Algorithmen nutzen den Stromspeicher, aber keiner übertrifft die Baseline.
UA-DQN benötigt weniger Interaktionen mit der Umgebung, um zu einer den DQN-Varianten überlegenen Strategie zu gelangen, erfordert aber mehr Rechenressourcen.

% LTeX: language=en-US
\newpage
\section*{Acknowledgements}
First and foremost, I would like to express my gratitude to Nicole Ludwig for being a very approachable, kind and helpful supervisor.
I would also like to thank Francesco Chini for sharing his knowledge of the dark art of Reinforcement Learning.

I am grateful to the entire MLSES group for providing such a welcoming and social atmosphere.
I want to thank my officemates and friends for keeping me sane througout the sometimes stressful phases of this project.
Finally, of course, I want to thank Toni for everything!

\cleardoublepage

%%%%%%%%%%%%%%%%%%%%%%%%%%%%%%%%%%%%%%%%%%%%%%%%%%%%%%%%%%%%%%%%%%%%%%%%%%%%%
%%% Table of Contents
%%%%%%%%%%%%%%%%%%%%%%%%%%%%%%%%%%%%%%%%%%%%%%%%%%%%%%%%%%%%%%%%%%%%%%%%%%%%%

\renewcommand{\baselinestretch}{1.3}
\small\normalsize

\tableofcontents

\renewcommand{\baselinestretch}{1}
\small\normalsize

\cleardoublepage

%%%%%%%%%%%%%%%%%%%%%%%%%%%%%%%%%%%%%%%%%%%%%%%%%%%%%%%%%%%%%%%%%%%%%%%%%%%%%
%%% List of Figures
%%%%%%%%%%%%%%%%%%%%%%%%%%%%%%%%%%%%%%%%%%%%%%%%%%%%%%%%%%%%%%%%%%%%%%%%%%%%%

% \renewcommand{\baselinestretch}{1.3}
% \small\normalsize

% \addcontentsline{toc}{chapter}{List of Figures}
% \listoffigures

% \renewcommand{\baselinestretch}{1}
% \small\normalsize

% \cleardoublepage

%%%%%%%%%%%%%%%%%%%%%%%%%%%%%%%%%%%%%%%%%%%%%%%%%%%%%%%%%%%%%%%%%%%%%%%%%%%%%
%%% List of tables
%%%%%%%%%%%%%%%%%%%%%%%%%%%%%%%%%%%%%%%%%%%%%%%%%%%%%%%%%%%%%%%%%%%%%%%%%%%%%

% \renewcommand{\baselinestretch}{1.3}
% \small\normalsize

% \addcontentsline{toc}{chapter}{List of Tables}
% \listoftables

% \renewcommand{\baselinestretch}{1}
% \small\normalsize

% \cleardoublepage

%%%%%%%%%%%%%%%%%%%%%%%%%%%%%%%%%%%%%%%%%%%%%%%%%%%%%%%%%%%%%%%%%%%%%%%%%%%%%
%%% List of abbreviations
%%%%%%%%%%%%%%%%%%%%%%%%%%%%%%%%%%%%%%%%%%%%%%%%%%%%%%%%%%%%%%%%%%%%%%%%%%%%%

% can be removed
% \addcontentsline{toc}{chapter}{List of Abbreviations}
% \chapter*{List of Abbreviations\markboth{LIST OF ABBREVIATIONS}{LIST OF ABBREVIATIONS}}

% \begin{tabbing}
% \textbf{FACTOTUM}\hspace{1cm}\=Schrott\kill
% \textbf{BLAST}\>Basic Local Alignment Search Tool \\
% \textbf{...} \> ...\\
% \end{tabbing}

% \cleardoublepage

%%%%%%%%%%%%%%%%%%%%%%%%%%%%%%%%%%%%%%%%%%%%%%%%%%%%%%%%%%%%%%%%%%%%%%%%%%%%%
%%% Der Haupttext, ab hier mit arabischer Numerierung
%%% Mit \input{dateiname} werden die Datei `dateiname' eingebunden
%%%%%%%%%%%%%%%%%%%%%%%%%%%%%%%%%%%%%%%%%%%%%%%%%%%%%%%%%%%%%%%%%%%%%%%%%%%%%

\pagenumbering{arabic}
\setcounter{page}{1}

%% Introduction
%%%%%%%%%%%%%%%%%%%%%%%%%%%%%%%%%%%%%%%%%%%%%%%%%%%%%%%%%%%%%%%%%%%%
% Einleitung
%%%%%%%%%%%%%%%%%%%%%%%%%%%%%%%%%%%%%%%%%%%%%%%%%%%%%%%%%%%%%%%%%%%%

\chapter{Introduction}\label{Introduction}
%- Because of climate change we switch the grid to renewable energies
Climate Change is the global challenge of our lifetime. Carbon introduced into the atmosphere when burning fossil fuels for human needs causes global warming, destabilizing the climate, ecosystems, and societies around the world.
In the Paris Agreement of 2015, governments have committed to an ambitious goal of drastically reducing carbon emissions to keep global warming from increasing beyond 2 °C, compared to 1990.
The latest report by the Intergovernmental Panel on Climate Change  urges governments to take stronger actions, or their previous commitment will not be reached \citep{portner2022ClimateChange2022}.
A key strategy for reducing carbon emissions from a range of sources is the combination of two measures: The first step is to electrify current processes that use fossil fuels, like replacing gas-fired furnaces with heat pumps. The second step is to replace carbon-intensive electricity generation with renewable options like solar and wind power.

%- renewable energies are less flexible than conventional energy supply
While much better for the natural environment, renewable sources of energy pose a challenge for a grid built for fossil fuels:
Unlike fossil-fuelled power plants, renewable power production depends on the weather, and it can not react flexibly to changes in demand.
As the share of installed renewable sources of electricity continues to grow, the reliability of the electricity supply will go down.

%- for a stable system, we need flexibility somewhere: supply, demand, or grid storage
In order to keep the grid stable, supply and demand must always be in balance. Before the green transition, this was achieved by flexible power generation: When demand was high, electricity producers were able to react and increase production.
This was incentivized by a complex and tightly regulated electricity market.
%    - supply doesn't work, so we need flexibility in demand and/or grid storage.
As the share of renewable power increases, fossil-fuelled electricity producers remain the only market participants that can flexibly react to changes in renewable production and demand for electricity.
When phasing out fossil-fuelled power generation, this flexibility needs to be provided by different parts of the system.

%    - in this thesis, I focus on demand-side flexibility.
There are dedicated electricity storage facilities that can react very quickly to stabilize the grid by storing and releasing energy as needed.
Grid-scale storage in Europe mainly consists of hydropower, which has been installed where mountainous geography allows, and capacity is limited \citep{gimeno-gutierrez2015AssessmentEuropeanPotential}.
More expensive battery-powered storage facilities are slowly being built, but are largely not cost-efficient in the current economic setting.
Therefore, there is a need for \emph{Demand Response}, or the ability of demand to flexibly respond to available electricity supply.

%- buildings are responsible for x\% of energy use
%- it makes sense for buildings to be able to react flexibly to changing conditions
Buildings use energy mainly for heating and cooling air and water supply.
Today, they are responsible for 17.5\% of global carbon emissions \citep{ritchie2020COGreenhouseGas}.
On the other hand, buildings often contribute to electricity production through photovoltaic panels. New buildings are often equipped with a battery, which enables them to more efficiently use their solar electricity.
Building electricity consumption is already largely automated and is therefore a prime candidate for automated demand response.

%- I use RL, an adaptive control algorithm, to provide this demand-side flexibility
%- cite RL in Demand Response
The \emph{Reinforcement Learning} (RL) family of control algorithms has shown promising performance in many control tasks, most notably video games (for example \cite{mnih2015HumanlevelControlDeep}), robotics \citep{latifi2020ModelFreeControlDynamicField} or wind power generation\citep{zhang2020ResearchAGCPerformance}.
RL is also being studied for controlling energy systems in buildings.
\cite{vazquez-canteli2020CityLearnStandardizingResearch} introduce CityLearn, which proved a popular simulation framework for applying RL algorithms to building energy storage control. Evaluated algorithms include Soft Actor-Critic \citep{pinto2021CoordinatedEnergyManagement} and DQN \citep{schreiber2020ApplicationTwoPromising}.

\cite{wang2020ReinforcementLearningBuilding} find that the demand for data is a key challenge for applying RL for building control.
\cite{clements2020EstimatingRiskUncertainty} introduce \emph{Uncertainty-Aware Deep Q-Networks} (UA-DQN), a distributional RL algorithm based on DQN \citep{mnih2015HumanlevelControlDeep} that uses explicit uncertainty estimates for approximate Thompson sampling.
On a video game task, they show that this allows the algorithm to explore more efficiently, reducing the need for data.
UA-DQN's explicit treatment of uncertainty should lead to overall better performance with less need for data, better robustness for novel data, and can even be leveraged for differently risk-aware charging and discharging strategies, all of which are important for building energy management.

In this thesis, I set out to test the performance of UA-DQN on a building battery control task implemented with CityLearn.
I compare its performance against two variants of DQN and a rule-based baseline.

%- my contribution is to apply uncertainty-aware RL, in the hopes of getting
%    - better performance 
%    - more robustness 
%    - risk-aware strategies
%- this contributes to our goal (flexibility in demand side).
This thesis is structured as follows: In chapter \ref{chap:background}, I motivate in more detail the need for Automated Demand Response. I introduce the theory of Reinforcement Learning and lay the foundations for the uncertainty-aware algorithm.
In chapter \ref{approach}, I present the uncertainty-aware algorithm, as well as a detailed description of the experimental setup.
I present the results in chapter \ref{results}.
A discussion and a short outlook conclude the thesis.

\cleardoublepage

%% 
%%%%%%%%%%%%%%%%%%%%%%%%%%%%%%%%%%%%%%%%%%%%%%%%%%%%%%%%%%%%%%%%%%%%
% Grundlagen
%%%%%%%%%%%%%%%%%%%%%%%%%%%%%%%%%%%%%%%%%%%%%%%%%%%%%%%%%%%%%%%%%%%%

\chapter{Background}
  \todo[inline]{Give all the necessary background that is needed to justify and understand the problem and the approach.
  - comment on employed hardware and software
  - describe methods and techniques that build the basis of your work
  - review related work(!)
  - roughly 1/3 of thesis
  }

% \todo[inline]{
%     - flesh out beginning from introduction, support general argument

%     1. supply side rigidity (plot of different energieträger over time)
%     2. status quo demand side rigidity
%     3. ansatzpunkte für Flexibilität in Demand Side mit RL oder so, konkrete Beispiele
    
%     4. RL ...

% }

\section{The Electrical Grid and Flexibility}
% \done[inline]{
%     1. Describe the goals and the structure of the Electrical Grid
%         - Provide reliable energy to consumers and industry
%         - Describe the changes caused by the energy transition
%         - History: Only fossil fuels and Water has been used, later nuclear.
%     2. Describe Challenges posed by the energy transition
%         - Electrification of processes causes more load, but building more transmission lines is costly
%             - But Cars can also be used as flexible electricity storage
%         - Specifically, last-mile grids are not built to handle peak solar production (is this true?)
%         - Phasing out reliable fossil fuel and nuclear plants causes less reliability
%         - Building renewable energy plants causes intermittent and unreliable energy supply
%         - Give number for how much flexibility we need (maybe ?)
%         - Grid storage
%             - Spinning reserve
%             - Pumped Water Storage
%             - Large-scale Battery storage systems
%             - Hydrogen
%             --> all together, not enough (support claim!)
%     3. Demand Side Flexibility
%         - Incentive-based demand reduction in industry
%             - (shutoffs in case of severe grid instability)
%         - Incentive-based demand shifting
%             - Shifting industrial processes
%             - private consumers (day/night-tariffs, flexible pricing)
%                 - Challenge: Electricity is cheap and reliable, incentives need to be very large, mental load
%                     - Consumers are unflexible in their use of appliances like kitchen stoves and lighting.
%                     - examples for incentive programs!
%                 - solar producing prosumers already sometimes care. (because of the "pricing model". They get their own solar power at cost, basically for free.)
%         - Smart Electrification makes more flexibility and efficiency available:
%             - Electric cars and home batteries can intelligently decide when to charge
%             - Electric (heat-pump) Heating and cooling can somewhat flexibly happen when electricity is cheap
%             - This amounts to a control problem and a coordination problem.
%             - Mention ways in which this problem has been tried to solve:
%                 - Rule-based controllers
%                 - Market mechanisms
%                 - RL
%                 - ...?
% }

% 1. Describe the goals and the structure of the Electrical Grid
The electrical grid is basic infrastructure that enables the function of our modern society.
It connects electricity consumers, from private consumers to heavy industry, with producers.
Historically, the entire system has been built for reliable large-scale power generation, overwhelmingly fuelled by coal and natural gas, together accounting for 67\% of Germany's electricity consumption in 1985, with 27\% provided by nuclear energy \cite{ritchie2022Energy}.

Climate change and the exit from nuclear power require a radical increase in the share of renewable electricity. As of 2021, renewable energy accounts for 40\% of electricity consumption in Germany \cite{ritchie2022Energy}.
An overview of the historical development is shown in figure \ref{fig:electricity_mix}

\begin{figure}
    \centering
    \frame{\includegraphics[width=\figurewidth]{figures/electricity_mix.png}}
    \caption{The share of renewable electricity has risen significantly from 1985 to 2021, while nuclear and fossil fuels have declined.}
    \label{fig:electricity_mix}
\end{figure}

While conventional energy generation can flexibly respond to demand, renewable energy depends on the weather.
It is therefore intermittent and harder to predict, see figure \ref{fig:flexibility}. To be able to meet demand, there is a new need for flexible backup power.
Natural gas plants are more environmentally friendly than coal plants and can be flexibly turned on and off in a matter of minutes.
As a result, the share of electricity powered by natural gas has risen along with renewables.

\begin{figure}
    \centering
    \includegraphics[width = \figurewidth]{figures/flexible_renewables.png}
    \caption{Renewable electricity is intermittent and hard to predict precisely. As the share of renewable electricity increases, other sources of electricity need to respond flexibly to meet demand.}
    \label{fig:flexibility}
\end{figure}


% 2. Describe Challenges posed by the energy transition
Another aspect of the energy transition is the electrification of many processes which previously used fossil fuels directly.
Internal combustion engine cars are being replaced by more efficient battery-powered electric cars and heating units that use natural gas or oil are being replaced by much more efficient electric heat pumps.
Many heat pumps can also be run in reverse to cool a building, which climate change makes more and more necessary, but this enables additional electricity demand in the summer.

The ongoing energy transition puts a lot of pressure on the grid.
On the one hand, the electricity supply is becoming less reactive, while on the other hand, many polluting processes are being electrified, causing additional demand.\todo{maybe give specific number}
During large future demand spikes, the total load on the grid could exceed current capacity. To cope, new transmission lines are being built, a costly and complicated process.

In order to both reduce reliance on fossil fuels and to keep the total load below grid capacity, there is a need for additional flexibility in the system.
A traditional source of flexibility has been pumped hydropower storage.
When there's unused electricity, it can be stored by pumping water uphill.
When needed, water can be released and used to generate electricity. 
Hydropower depends almost entirely on geography. There is almost no potential to develop more hydropower storage.

A different large-scale method of energy storage is batteries, which have only found limited use due to their cost. Hydrogen can also be used as a way to store and even transport energy, with several chemical processes currently being explored.
However, this method is also costly, and any produced hydrogen is probably worth more as a crucial chemical than as pure electricity storage.\todo{rephrase, cite}

As flexibility in electricity supply decreases, and centralized storage facilities remain too expensive, electricity demand needs to become more flexible.


\section{Demand Response}
% 3. Demand Side Flexibility
% \todo[inline]{       
% - Demand side flexibility is a good idea.
% - what does demand respond to?

% - Incentive-based demand reduction in industry
% - (shutoffs in case of severe grid instability)
% - Incentive-based demand shifting
% - Shifting industrial processes
% - private consumers (day/night-tariffs, flexible pricing)
%     - Challenge: Electricity is cheap and reliable, incentives need to be very large, mental load
%         - Consumers are unflexible in their use of appliances like kitchen stoves and lighting.
%         - examples for incentive programs!
%     - solar producing prosumers already sometimes care. (because of the "pricing model". They get their own solar power at cost, basically for free.)
% - Smart Electrification makes more flexibility and efficiency available:
% - Electric cars and home batteries can intelligently decide when to charge
% - Electric (heat-pump) Heating and cooling can somewhat flexibly happen when electricity is cheap
% - This amounts to a control problem and a coordination problem.
% - Mention ways in which this problem has been tried to solve:
%     - Rule-based controllers
%     - Market mechanisms
%     - RL
%     - ...?
% }
Both the inflexibility of renewable power generation and the limited capacity for centralized flexible energy storage leave one component of the electric system: In a renewable-dominated grid, demand needs to flexibly respond to changes in available supply.
In order to stabilize the grid, grid operators employ schemes to curb demand in case of exceptionally large pressure on the grid.
Large industrial consumers are paid in advance for shutting off their processes if needed.
As a matter of last resort, rolling blackouts are introduced to curb demand when electricity production can not keep up with consumption.
An electric grid designed for renewable energy needs more fine-grained coordination across a larger fraction of demand.

In order to implement demand response, electricity consumers need to be incentivized and able to adapt their processes to available supply.
Proposed incentive schemes for demand response include market-based solutions that feature a flexible electricity price and solutions of centralized control that pay out flat rewards for participation, as well as combinations of the two. \todo{cite}
In order to be able to react to changes in demand, electricity-consuming processes need to be aware of the current and future available supply.
This can be a human in the loop, deciding to shift a process to a time with cheaper electricity, or this can be automated.
For example, a private prosumer that produces solar electricity might prefer to run their washing machine only on sunny days when electricity is free to them.
However, this means the human needs to keep track of the weather and is put under additional cognitive load when planning this.

As stated before, the process of heating is being electrified.
While this increases the load on the electric grid, it is also a chance to provide flexibility:
When equipped with an intelligent and connected control system, the heating system can flexibly react to changes in price, using hot water tanks or the building itself as heat storage.
Similarly, when intelligently automated, the charging process of an electric car or a home battery can somewhat react to market conditions.
\todo{somehow mention coordination goals}

Technologically, this amounts to a control problem:
The controller's goals are to meet certain demands (like ensuring a comfortable living temperature) while minimizing cost. In the context of the larger system, there is the shared goal of coordination between different electricity-consuming processes.
This involves both an understanding of the dynamics of the controlled system and an understanding of how prices are likely to change.
Different technological approaches can be employed for this.
When system dynamics are known and prices change predictably, for example with fixed rates per time of day, an optimal rule-based controller can be derived. \todo{cite and make sure this is true!}
A rule-based controller has the advantage of being transparent and reliable, but it's not flexible enough to react to changing system and pricing dynamics.
Several adaptive control algorithms have been proposed for use in demand response. \todo{List some and cite}
\missingfigure{illustrate control problem}
\todo[inline]{
    notes on this section:
        - focus more on buildings
        - use the following review paper:
        - \cite{li2021EnergyFlexibilityResidential}, a review paper about energy flexibility in residential buildings
}


\section{Reinforcement Learning}

\todo[inline]{
    1. Explain RL Fundamentals
        1. Markov Decision Process
            1. Highlight stochasticity of transition and Reward functions
            2. Reward function?
            3. Mention generalizations: POMDP, (Multi-Agent MDP?), Non-stationary MDP
        2. Value Learning
            - Introduce the notion of a state value
            - Bellman updates
            - Q-Learning
            - Mention Policy Gradient methods
        3. Deep Q-Learning
            - introduce Fundamentals of Deep Learning? Deep RL is somewhat like using a POMDP?
            - Introduce further tricks employed by DQN
                - Replay Buffer
                - Dueling Q-Networks?
                - TD-Learning / Target Network
        4. Importance of Action Selection and Exploration strategies
            - Exploration vs. Exploitation Dilemma
            - Convergence guarantees
}


% 1. RL Fundamentals
\subsection{Reinforcement Learning Fundamentals}
Reinforcement Learning (RL) is a feedback-based learning paradigm derived from behavior learning in animals.
This section serves as a brief introduction to the topic.
Unless otherwise stated, this section is based on the textbook \cite{sutton2018ReinforcementLearningIntroduction}.
In Reinforcement Learning, there is a clear distinction between the learning agent and the environment.
The agent is able to observe the state of the environment and perform an action.
In turn, the environment is affected by the action and transitions into a new state according to its stochastic transition dynamics.
The environment passes the resulting state and a reward signal back to the agent.
Typically, the agent's goal is to select actions that obtain the maximum reward.
In an infinite environment, the objective is to maximize the expected value of the total discounted future reward.

The environment specifies the entire reinforcement learning task.
Formally, it is a discounted Markov Decision Process (MDP):
$$ \text{MDP} = (S, A, R, \gamma, p),$$
where $S$ is the set of possible states, $A$ is the set of possible actions, $R$ is the set of possible rewards, $\gamma$ is the discount rate and $p(s', r|s,a)$ is the probability distribution that specifies the environment dynamics.
It is important to note that an MDP has the Markov Property, i.e. the dynamics depend entirely on the state and action, there is no hidden state.
The state space can therefore also be called the observation space.

When modeling a real-world control problem, an MDP necessarily is a simplifying assumption.
In reality, state transitions often depend on outside influences or are non-stationary for other reasons.
In complex problems, the desired behavior is not obvious. Therefore, designing the reward function is often non-trivial.

\todo{Mention a few RL applications that can be used as examples}

% 2. Towards DQN
% - Value Learning
Some environment states are preferable to others.
Value Learning is a class of RL algorithms that builds on this intuition.
From a trajectory of state-action-observation-reward tuples that were generated while following a policy $\pi$, a Value Learning algorithm assigns each state the expected future discounted reward that will be received when the agent, in state $s$, continues to act according to $\pi$:

$$v_\pi(s) = \mathbb{E}[G_t | S_t=s] = \mathbb{E}[\sum_{k=0}^{\infty}\gamma^kR_{t+k+1} | S_t = s], \text{for all } s \in S$$

Similarly, one can define the value of taking an action in a certain state:
$$ q_\pi(s,a) = \mathbb{E}[G_t | S_t=s, A_t = a] = \mathbb{E}[\sum_{k=0}^{\infty}\gamma^kR_{t+k+1} | S_t = s, A_t = a]$$

This Action-Value Function or Q-function induces a greedy policy by evaluating $q(s,a)$ for the current state and all possible actions, and selecting the highest-expected reward action.
For the optimal policy $\pi_*$, the greedy policy induced by $q_{\pi_*}$ is $\pi_*$ itself.
For more detail on how the optimal policy can be approximated, please refer to \cite{sutton2018ReinforcementLearningIntroduction}.

%explain (e-greedy) action selection
To iteratively improve an initial Q-function, the greedy policy induced by it can be used to generate new trajectories to learn from.
It intuitively serves as an approximation to $\pi_*$.
However, a slight modification needs to be made: To make sure all states are visited, the $\epsilon$-greedy policy is used instead, which acts randomly with a small probability of $\epsilon$, and acts greedily otherwise.

% - Deep Q-Learning
\subsection{Deep Q-Networks}
\cite{mnih2015HumanlevelControlDeep} propose the Deep Q-Network algorithm (DQN), which is able to learn and play many video games better than humans.
DQN is a Q-Learning algorithm.
It approximates the Q-function using a deep convolutional neural network, which is able to store complex information like video game dynamics.


In order to arrive at an efficient algorithm that shows stable performance in practice, they employ additional tricks.
Firstly, a replay buffer stores past trajectories

- Experience Replay
- Target Q-Network


This progress stems mainly from the ability of neural networks to accurately model highly complex functions like environment dynamics.


\todo[inline]{
    2. Related Work
        1. Uncertainty-Aware Reinforcement Learning: Related Work
            - Uncertainty-Aware Machine Learning Methods: Distributions instead of estimators
                - Explicitly model input uncertainty.
                    - Assume a distribution, model parameters
                    - e.g. Discrete distribution, normal distribution, beta distribution, or general distribution parameterized by quantiles.
                    - Example in RL: POMDP
                    - Algorithms: Bayesian NN, Gaussian Processes, Variational Autoencoders (uncertainty about deep representation)
                    - Advantages: More robust
                    - Disadvantages: More expensive, need to model data generation process
                - Explicitly model output uncertainty.
                    - In classification: Confidence score (discrete distribution), class probabilities
                    - In regression: Again, model distribution and parameters
                    - Examples: Bayesian NN, Ensemble models, Distribution Learning, *UA-DQN*
                    - Advantages: More trustworthy and more useful for risk-aware decision-making, more robust, can be more sample-efficient
                    - Disadvantages: more expensive, more complex, less robust
            - Examples of Uncertainty-aware RL applications:
                - TODO
        2. Reinforcement Learning for Consumer Building Demand Response
            1. Mention Frameworks, available data, including CityLearn, real-life applications
            2. Mention related work: applied algorithms and their results
                - 
}


































\clearpage
------- Erster Entwurf below ------

\subsection{Flexibility (/Demand Response)}
\todo[inline]{This section introduces the notion of flexibility and therefore motivates demand response.
It should also maybe talk about conflicts of interest, and incentives.

need to be defined before this section:
- conventional power
- renewable power
- supply/demand vs production/use}


- flexibility is needed both in the market layer and in the physical layer

Renewable electricity is less flexible than conventional electricity production, therefore there is a need for more flexibility elsewhere in the system.
Electricity production and electricity use must always match. If there is more or less electricity used than is available, the entire grid is no longer stable.
Partly, this coordination problem between producers and consumers is solved by the electricity market:
An electricity producer will produce and feed to the grid only as much electricity as they can sell.
This system works fine if the electricity producer can freely regulate how much power they produce based on the market.

This is not true for renewable energy, where production capacity is both determined by external factors and harder to predict.
Renewable sources of electricity are therefore less flexible.
In order to keep production and consumption in balance, there needs to be a system component with enough flexibility to respond to sudden changes in electricity supply and demand.
This can be a hydroelectric or grid-scale battery storage system which can both produce and consume electricity.
It can also be a conventional power plant that's kept in spinning reserve\todo{define}, to take over when there's less renewable electricity than needed. This causes carbon emissions.
Often, renewable power plants can't feed all of their production into the grid, because they can't find a flexible buyer for unexpected production.
This motivates the need for flexibility on the demand side.



\subsection{Demand Response}
- a scheme to operate the grid more efficiently as demand rises quickly.
- breaks with one of the central guarantees: electricity is freely available, accepting that renewable energy is much less reliable.
- Mechanism: Incentivize consumers to use electricity when there's capacity.
- implementations: TODO examples (Large-scale AC cuts, incentive programs, etc.)
- already in place for large industrial consumers who buy electricity on the markets. (TODO: Is that true?)
    - some industrial consumers are paid for providing flexibility, they can stop their processes if there's not enough electricity being produced.
- more difficult for private consumers, who don't want to think about efficiency all the time. Solution: Automate where possible.


\section{Reinforcement Learning}

\subsection{Fundamentals: Markov Decision Process}
Many animals are able to learn complex behaviors by performing an action, observing the results and - if the action got them closer to their goal - repeating the action when facing a similar situation.
For example, consider a food-dispensing lever in a hamster's cage. At first, the hamster might not notice the lever. But sooner or later, by accident or out of curiosity, the hamster will push the lever, dispensing an item of food. After a few repetitions, the hamster will know to walk towards the lever, and push the lever when it wants food.
Our hamster uses a biological implementation of \textit{Reinforcement Learning (RL)}.
In an RL environment, an \textit{agent} is able to repeatedly perform an \textit{action}, observe the consequences and be rewarded or punished.
RL therefore is an adaptive approach to solve feedback-based control problems.

Formally, RL environments are \textit{Markov Decision Processes (MDP)}:
$$ \text{MDP} = (S, A, p)$$
where $S$ is the set of possible states the environment can be in, $A$ is the set of actions available to the agent,
and the transition distribution $p(s', r \mid s, a)$ specifies the environment dynamics, giving the joint probability of transitioning from state $s \in S$ to state $s'\in S$ after performing action $a \in A$, and getting reward $r \in \mathbb{R}$.

Interacting with the MDP, an agent first observes the environment's state $s$, then takes an action $a$, and then the next state $s'$ of the environment is sampled from $p$.
There are no further restrictions placed on the structure of state and action spaces. They can be merely labelled or ordered, finite or infinite, single- or multidimensional.
Usually the agent observes a collection of variables and chooses an action along one or several dimensions.

Crucially, the states have the \textit{Markov Property}: The probability of transitioning to state $s'$ only depends on the current state $s$ and action $a$.
The history of past states does not matter, and there are no hidden facts that could change the probabilities.
In practice, this is a simplifying modelling assumption.
Returning to our hamster: In reality, the food-dispensing mechanism runs out of food after being activated a number of times. However, modelling the situation as an MDP, we do not keep track of history, so we can't tell in advance whether pushing the lever will actually produce food. Instead, we assume there's a certain probability of the action being unsuccessful, that is $p(s', \text{food} \mid s, \text{lever pushed}) < 1$.
One generalization of the MDP that does enable us to model such hidden variables is the Partially Observable MDP, which I will cover in more detail later on \todo{cite section}.

\subsection{Reinforcement Learning}
\todo[inline]{introduce RL terms and algorithms}

Let's now turn to the question how to solve a Markov Decision Process. Solving an MDP usually means finding the sequence of actions that maximize the expected total reward over all future time steps.

todo:

- episode
- policy
- value function
- exploration vs exploitation
- Bellmann Update


Baseline algorithm: \todo{which one? depends on my treatment of uncertainty in the other one, so on the approach}
- One RL algorithm, which I use as a baseline, is ???
- Explain algorithm



\section{Uncertainty in Reinforcement Learning}
\todo[inline]{Introduce Uncertainty terms and formalisms from different perspectives. Then apply to RL.}
There is a rich body of work on uncertainty. Mathematical and statistical notions of uncertainty, perspectives from economics for decision making under uncertainty.

Notes:

- Motivation: Most information is uncertain to some extent. Making good decisions under uncertainty requires an awareness of the uncertainty.
- Uncertainty vs Risk: Uncertainty is a measure of the information content of a random variable or an observation???, Risk is the cost associated with different situations.
- formal framework (maybe borrow from Econ: When to buy or sell a given asset?)
- Decision making under uncertainty
    - which objective (Expected value vs risk metrics)
- different types of uncertainty (e.g. aleatoric vs epistemic)
    - there are different types of uncertainty: Some uncertainty can be reduced by learning more about the problem, other uncertainty can not.
    - this stems from the formulation of RL as a stochastic MDP
    - for example, a biased coin. You will be able to learn something about it, but not actually predict the outcome ???

Uncertainty in RL: (maybe this should already be in approach?)
- How is Uncertainty commonly modelled?
    - epistemic uncertainty in the observations: not explicitly modelled, somewhat represented in state-value function
    - stochasticity in the environment dynamics (+consequences of actions): accepted in the MDP. learned as transition probabilities in model-based RL, subsumed in e.g. Q-function in model-free RL.
    - epistemic uncertainty in the environment dynamics: modelled as transition probabilities
    - stochasticity in the reward function: not usually explicitly modelled
    - epistemic uncertainty in the reward function: modelled implicitly in the state-value function
    - uncertainty about causality? - probably not really relevant? should I discuss it somewhere else?
- Adaptations for explicit treatment of uncertainty:
    - Formalism for non-perfect observations: POMDP (usually there are hidden variables) -> Usually solved by estimating an MDP, solving that.
        - POMDP does not assume the Markov property on observations, but does assume a hidden MDP
    - ???
    - RL from human preferences? (for learning a reward function)
- Benefits of explicit treatment of uncertainty/Motivation:
    - Risk-aware strategies
    - better performance (maybe? TODO: Test this)
    - more robustness (possibly? TODO: support this or not)
    - better interpretability (possibly?)
    - TODO: other
- Drawbacks:
    - more complex models require more training data
    - less efficient algorithms
    - not as well understood theoretically
    - might perform worse than just learning everything implicitly!
    
Risk-aware strategies:
    - can either specify a risk tolerance at time of inference or during training
    - during training: change reward function
    - at time of inference: requires model of the environment (I think) or a Q-function
    - more robust: can hand over control to e.g. humans when uncertain
    
Uncertainty in Multi-Agent Learning: (maybe exclude this completely)
    - Multi-Agent Environments are characterized by simultaneous actions by multiple agents, who each learn and act according to their own rewards.
    - More realistic and resilient than centralized control
    - absent trust, might be stuck in a suboptimal equilibrium



  
\cleardoublepage

%% 
\chapter{Approach}
    \label{approach}
    
\todo[inline]{
Roughly 1/3 of thesis}
\done[inline]{- restate the research question, and how I set out to answer it.


Plan for this chapter:
- introduce CityLearn and motivate the choice
    - mention citylearn challenge.
- Introduce and explain the UADQN algorithm and motivate the choice (other algorithms have been tried)
- Explain Experiment Setup, High-level overview to implementation details each.
    1. Hand-Engineered Benchmark agent and Discretization Experiment
    2. Hyperparameter Tuning Setup:
        - High-Level Overview of Experiments
        - Design decisions of Tuning Experiment
        - Implementation details (all the way down to hardware)
    3. Test performance of Tuned Agents
        - High-Level Overview of Experiment, motivated by story
        - Investigate performance of Tuned Agents
            - High-Level Overview of Investigation

}





In order to efficiently control electricity demand, one needs to act with foresight on uncertain information.
I aim to find out whether existing approaches can benefit from explicitly modelling various uncertainties.
In order to do so, I introduce CityLearn, a sample control environment and model its uncertainties. I model CityLearn from different perspectives.
Drawing from the theoretical background, I then construct an uncertainty-aware RL model to solve CityLearn.
I compare this uncertainty-aware model empirically against existing baselines.
An uncertainty-aware model can have additional benefits, like risk-awareness. I explore these. 

\todo[inline]{- Divide it into subquestions, and treat each subquestion separately.}

\section{RL for Demand Response}
- Let's focus in on Automated Demand Response in buildings:
    - Mention incentive-based behavior-change models, but ML models can do better!
    - Define the setting:
        - Focus on buildings, they need lots of energy and the physical processes are comparably simple, standardized and predictable.
        - Ignore industrial DR programs.
        - decentralized (why?)
- What is the state of the art for Automated Demand Response?
- How else can we implement Automated DR?


- Introduce CityLearn and alternatives


\subsection{CityLearn: Applying RL for Demand Response}
\todo[inline]{This section of the approach/background introduces the CityLearn Framework, so I can later mention it. It motivates the choice while highlighting alternative options. It introduces the data, modelling limitations and setting, but not the exact experimental setup.}

CityLearn as a model for building-based demand response:
    - Model description (data, available observations and actions, rationale)
    - Success measures, cost function design
    - CityLearn challenge
    - limitations: modelling errors/simplifications and how much they matter
        - CityLearn was built to assess the general usefulness of RL for DR, not for my specific question.
        - Battery efficiency
        - Weather Forecasts are perfect oracles
        - Only available action is battery charging/discharging.
        - Other processes are assumed to be fixed (at least in 2022 challenge). Automatic Washing Machine starting.
        - Human behavior can not be influenced (models humans as inflexible)
        - Hourly control instead of continuous -> Makes perfect control more difficult
        - Energy can not be sold to the grid, not even to the microgrid. -> unrealistic
        - 2022 challenge: initially did not encourage cooperation.
    - Strengths: For what tasks is this environment adequate?
        - Makes it possible to apply RL to DR without expensive computational overload*
        - Test Multi-Agent behavior and cooperation in the DR context, as opposed to only single-building frameworks
        - Enables comparison of different approaches as a benchmark
    - Formal treatment of the CityLearn environment
        - Why? What kind of formal treatment?
    
Alternatives to CityLearn
    


\section{Theoretical approach}
\subsection{Setting the scene}
\todo[inline]{- start with a theoretical approach:
    - formal model of the environment,}
In order to develop our uncertainty-aware approach to demand side management, we need to have a good understanding of the information environment.
Let's formalize all relevant goals, available information and uncertainties.

The idea of Demand Side Management was introduced to enable a more efficient usage of the grid.
Depending on who implements Demand Side Management, different incentives and information infrastructure might evolve.
In our model of demand response, consumers themselves are empowered to make decisions about their electricity demand.


\todo[inline]{    - !!! where does uncertainty come from in CityLearn !!!,}
- forecasting problems:
    - uncertainty about future occupant behavior (electricity demand)
    - uncertainty about future solar power production (weather forecasts)
    - uncertainty about future costs (price forecasts)
- measurement uncertainty:
    - observations might be imprecise
- coordination uncertainty:
    - uncertainty about other actors' strategy and current actions
- reward uncertainty:
    - uncertainty about the consequences of actions
    - the observed reward might be imprecise
    - the observed reward might not be the actual desired reward ???
    
Which uncertainties need to be specially treated?

\todo[inline]{What advantages would an uncertainty-aware strategy have?}
- better risk management (possibly)
- better performance (possibly)
- better interpretability and robustness (possibly)


\subsection{Developing the approach}
\todo[inline]{    - developing the algorithm (this is important. Maybe it includes some actual theoretical work and insight!),}
\todo[inline]{        - build off approaches that are cited in background.}
\todo[inline]{    - explain and motivate decisions.}

\section{Practical Approach}
\subsection{Implementation Details}
\todo[inline]{- high-level overview of the practical approach and setup}
- describe hand-engineered strategy

- Train one single agent that works for all buildings independently

Adaptations from the Risk and Uncertainty Paper algorithm:
- make it work with continuous action space (discretize)
- make it work for the single-agent case.

\todo[inline]{- describe implementation details
    - (of my algorithm and of CityLearn)
    - I found/fixed CityLearn Bugs
    - I engineered the Reward Function}
    
    
\subsection{Experimental Setup}
\todo[inline]{- Design/select an algorithm that is fit for the task. Run the experiment and investigate results (in separate chapter)}

- Data?
    - CityLearn.

- I want to test the performance of different strategies:
    - doing nothing as baseline
    - basic rule-based controller as improved baseline
    - various risk-aware strategies


\cleardoublepage

%%
%%%%%%%%%%%%%%%%%%%%%%%%%%%%%%%%%%%%%%%%%%%%%%%%%%%%%%%%%%%%%%%%%%%%
% Ergebnisse
%%%%%%%%%%%%%%%%%%%%%%%%%%%%%%%%%%%%%%%%%%%%%%%%%%%%%%%%%%%%%%%%%%%%
\chapter{Results}
  \label{results}

% \todo{
% In this chapter which also could be more than one chapter, depending on the nature of the thesis, the results of the thesis are presented.
% Make sure you illustrate your results with appropriate figures and tables, but do not discuss the results here. This should be done in a separate discussion chapter.
% Or maybe do combine results and discussion and split by research questions.
% }
\todo[inline]{
  Incorporate Nicole's feedback! (see slack DMs)
}

\section{Rule-Based Controller}
- How does it perform?
- On how many days does it needlessly fill the battery?

\section{Discretization}
\begin{figure}[h]
  \centering
  \includegraphics[width=\figurewidth]{figures/discretization.png}
  \caption{This graph shows the effect of the discretization resolution on the rule-based control algorithm. The coarsest resolution that does not significantly impact performance and includes the zero action is 9. In this graph, 0 possible actions means no discretization is applied.}
  \label{fig:discretization}
\end{figure}

In the pre-experiment on the effect of discretization, the rule-based agent performs comparably to the continuous case when discretization resolution is high, and worse on coarse resolutions. The coarsest resolution that performs similarly well as the continuous case is the subdivision into 8 or 9 possible actions.




\section{Hyperparameter Tuning}
\todo{
- How much computational power did the tuning take?
  - if I can find it. If not, just omit.
}
All 200 runs of $\epsilon$-greedy DQN completed successfully.
10/100 runs of UA-DQN, all runs with a batch size of 1, failed.
18/200 runs of DQN-softmax failed, all of which share a batch size of 8.
However, 17 other runs with a batch size of 8 completed successfully.

Table \ref{tab:hyperparameters} shows the results of the tuning process for each algorithm.
For each algorithm, the learning rate had the most significant correlation with final performance, followed by the batch size and the target network update frequency.
The $\epsilon$ parameter of the optimizer Adam showed a large effect only for $\epsilon$-greedy DQN.

Figure \ref{fig:tuning_results} shows a histogram of the final performance of all successful tuning runs.
Both DQN algorithms show a peak at around -5000, which corresponds to strategies that make no or very little use of the battery.
The UA-DQN algorithm learns a superior strategy for more hyperparameters.

\begin{table}[h]
  \centering
  \caption{This Table shows the correlation between tuned hyperparameters and algorithm performance for successful runs.}
  \label{tab:hyperparameters}
  \begin{tabular}{c||c|c|c}
    Algorithm & Hyperparameter & Value for Best Run & \makecell{Correlation with \\ final score} \\ \hline
    UA-DQN & Learning Rate                                & 3e-4 & \bf{-0.47}\\
           & Batch Size                                   & 128 &  0.28\\
           & Adam's $\epsilon$                            & 1e-07 & -0.03\\
           & \makecell{Target Network \\ Update Frequency}& 4 & -0.11\\ \hline
    DQN-softmax & Learning Rate                                 & 3e-4 & \bf{-0.36} \\
                & Batch Size                                    & 128 & -0.18\\
                & Adam's $\epsilon$                             & 1e-05 & -0.02\\
                & \makecell{Target Network \\ Update Frequency} & 4 &  0.16\\ \hline
    DQN-$\epsilon$-greedy & Learning Rate                       & 7e-05 & \bf{-0.33}\\
                & Batch Size                                    & 4 & -0.19\\
                & Adam's $\epsilon$                             & 1e-08 &  0.10\\
                & \makecell{Target Network \\ Update Frequency} & 16 &  0.14

  \end{tabular}
\end{table}



\begin{figure}
  \centering
  \includegraphics[width=\figurewidth]{figures/tuning_results.png}
  \caption{This histogram shows the final episode reward of all successful tuning runs.}
  \label{fig:tuning_results}
\end{figure}

\section{Comparison of Tuned Algorithms}
Figure \ref{fig:tuning_validation} shows the episode rewards of the selected hyperparameters for each algorithm during training.
Tuned UA-DQN converges faster than the other algorithms, and it converges to a better mean performance, as stated in table \ref{tab:tuned_results}. The hand-engineered rule-based agent outperforms the tuned reinforcement learning algorithms on both metrics.

When repeated with 10 different seeds, a single run of DQN-softmax failed, compared to no failures from the other algorithms.


\begin{figure}
  \centering
  \includegraphics[width=\figurewidth]{figures/tuning_validation.png}
  \caption{This graphic shows the performance during training of the three tuned algorithms. The shaded area shows the standard deviation over 10 runs with different random seeds. Tuned UA-DQN converges after fewer episodes than either DQN variant. UA-DQN was only trained for 50 episodes due to the added computational complexity of the algorithm.}
  \label{fig:tuning_validation}
\end{figure}

\begin{table}
  \centering
  \caption{This table shows the mean performance of tuned algorithms when evaluated using their respective action selection policy for one episode on Building 1.}
  \label{tab:tuned_results}
  \begin{tabular}{l|ccccc}
    Agent                 & Dollar Cost & Carbon Emission & Average \\ \hline
    Control (No Action)   & 1    & 1    & \textbf{1}    \\
    DQN-Softmax           & 0.83 & 0.93 & \textbf{0.88} \\
    DQN $\epsilon$-greedy & 0.82 & 0.93 & \textbf{0.88} \\
    UA-DQN                & 0.82 & 0.91 & \textbf{0.87} \\
    Discrete Rule-Based   & 0.80 & 0.88 & \textbf{0.84}
  \end{tabular}
\end{table}

\begin{figure}
  \centering
  \includegraphics[width=\figurewidth]{figures/non-greedy-fraction.png}
  \caption{This graphic shows the fraction of selected actions that do not correspond to the highest expected value. It highlights the different action selection strategies employed by the different algorithms. UA-DQN keeps exploring for longer than the other strategies.}
  \label{fig:non_greedy_fraction}
\end{figure}

To illustrate the difference in exploration between the algorithms, figure \ref{fig:non_greedy_fraction} shows the fraction of selected non-greedy actions per episode.
All algorithms start out exploring more and then gradually decrease their exploration rate.
DQN-softmax and UA-DQN explore more than $\epsilon$-greedy DQN, which quickly reaches an exploration rate of $\epsilon = 0.02$.
UA-DQN keeps exploring more than the other algorithms.

\clearpage

\cleardoublepage

%%
%%%%%%%%%%%%%%%%%%%%%%%%%%%%%%%%%%%%%%%%%%%%%%%%%%%%%%%%%%%%%%%%%%%%
% Diskussion und Ausblick
%%%%%%%%%%%%%%%%%%%%%%%%%%%%%%%%%%%%%%%%%%%%%%%%%%%%%%%%%%%%%%%%%%%%

\chapter{Discussion}
  \label{Discussion}

\todo[inline]{
Of course very important! You need to discuss the informatics as well as econ part of your thesis topic.

\bigskip
Take your time for writing the discussion, besides the introduction chapter it is the most important chapter of your thesis.
Also do not subsection the discussion too heavily.

At least 5 pages,

Outlook can become an extra chapter.
}

\todo[inline]{
  Outline/structure for this chapter:
  TODO: Fill with content

  - restate research question: Does an uncertainty-aware algorithm outperform other methods on this problem?
  - summarize key findings (answer question)
    - On this problem, UA-DQN learns faster than other methods. --> It explores more efficiently.
    - Its final performance is not significantly better than others (compute p value maybe, do t-test. Does that make sense?)
    - For a simple system, a tuned rule-based policy can outperform RL methods.
    - UA-DQN takes more computational resources.
  - explain and interpret results in more detail
    - relate back to other literature
    - support conclusions with data from results
    - ...
  - Discuss limitations
    - Internal Validity: Does the experiment support the conclusion?
      - Experiment was not thorough enough
      - Absence of good performance does not mean good performance is impossible, maybe I just did it wrong
      - Data was not fully used
    - External Validity: Does the conclusion generalize to real life?
      - Model limitations: Does CityLearn represent the real life application?
        - The real life application of such a system depends on technical and regulatory details
        - Data is real life, but other data might be available.
        - A real life application could have different goals than only price and CO2, like coordination goals and communication requirements.
      - 
    - Is the question important?
      - what should such a system even be judged on in application?
  - Further Research
    - Application in a real life system
    - 
}

\todo{mention that e-greedy epsilon was not tuned, but it definitely should have been!}
- Tuning: Initialization might have been different for different algorithms

- What would a good setting for the risk-aversion parameter even be?

\subsubsection{Known Limitations of the RL environment} %TODO maybe this should be in Discussion only?
- data: Weather forecasts are not real forecasts
- energy model: Battery charging speed is symmetric
- market model: No selling the electricity
- Coordination goals: Were introduced later in the challenge, not used here.
    - Only single building.

CityLearn as a model for building-based demand response:
- limitations: modelling errors/simplifications and how much they matter
    - CityLearn was built to assess the general usefulness of RL for DR, not for my specific question.
    - Battery efficiency
    - Weather Forecasts are perfect oracles
    - Only available action is battery charging/discharging.
    - Other processes are assumed to be fixed (at least in 2022 challenge). Automatic Washing Machine starting.
    - Human behavior can not be influenced (models humans as inflexible)
    - Hourly control instead of continuous -> Makes perfect control more difficult
    - Energy can not be sold to the grid, not even to the microgrid. -> unrealistic
    - 2022 challenge: initially did not encourage cooperation.
- Strengths: For what tasks is this environment adequate?
    - Makes it possible to apply RL to DR without expensive computational overload*
    - Test Multi-Agent behavior and cooperation in the DR context, as opposed to only single-building frameworks
    - Enables comparison of different approaches as a benchmark
\cleardoublepage

%%
%%%%%%%%%%%%%%%%%%%%%%%%%%%%%%%%%%%%%%%%%%%%%%%%%%%%%%%%%%%%%%%%%%%%
% Zusammenfassung
%%%%%%%%%%%%%%%%%%%%%%%%%%%%%%%%%%%%%%%%%%%%%%%%%%%%%%%%%%%%%%%%%%%%

\chapter{Conclusion and Outlook}
  \label{conclusion}

\todo[inline]{
- 1 page
- summarize again what your paper did, but now emphasize more the results, and comparisons
- write conclusions that can be drawn from the results found and the discussion presented in the paper
- future work (be very brief, explain what, but not much how)


Summary:
- I evaluate UA-DQN on a custom task using CityLearn.
- Tuned UA-DQN requires fewer environment interactions than variants of DQN to learn a good policy. This suggests that efficient exploration is beneficial, and exploration in general is useful for Q-Learning on CityLearn.
- Tuned UA-DQN does not outperform a simple rule-based policy.

Conclusions:
- UA-DQN is a promising algorithm for environments where exploration is costly.

Future Work:
- CityLearn with integrated control
- Coordination task
- Uncertain Weather forecasts.
- how does UA-DQN with a nonstationary environment? How much does a learned policy transfer to a different building, for example?
}

In this thesis, I apply the UA-DQN algorithm to a custom building energy management task implemented in CityLearn.
I find that tuned UA-DQN outperforms DQN in order to learn a good policy. This means things are harder to get right.




\cleardoublepage

%%%%%%%%%%%%%%%%%%%%%%%%%%%%%%%%%%%%%%%%%%%%%%%%%%%%%%%%%%%%%%%%%%%%%%%%%%%%%
%%% Appendix
%%%%%%%%%%%%%%%%%%%%%%%%%%%%%%%%%%%%%%%%%%%%%%%%%%%%%%%%%%%%%%%%%%%%%%%%%%%%%
\appendix

%\setcounter{secnumdepth}{-1}
%\section{Tables}\label{chap:App}
%\chapter{Further Tables and Figures}\label{chap:App}


%\cleardoublepage

%%%%%%%%%%%%%%%%%%%%%%%%%%%%%%%%%%%%%%%%%%%%%%%%%%%%%%%%%%%%%%%%%%%%%%%%%%%%%
%%% Bibliographie
%%%%%%%%%%%%%%%%%%%%%%%%%%%%%%%%%%%%%%%%%%%%%%%%%%%%%%%%%%%%%%%%%%%%%%%%%%%%%

\addcontentsline{toc}{chapter}{Bibliography}

\bibliography{mylit, papers}
%% Obige Anweisung legt fest, dass BibTeX-Datei `mylit.bib' verwendet
%% wird. Hier koennen mehrere Dateinamen mit Kommata getrennt aufgelistet
%% werden.

\cleardoublepage
%%%%%%%%%%%%%%%%%%%%%%%%%%%%%%%%%%%%%%%%%%%%%%%%%%%%%%%%%%%%%%%%%%%%%%%%%%%%%
%%% Erklaerung
%%%%%%%%%%%%%%%%%%%%%%%%%%%%%%%%%%%%%%%%%%%%%%%%%%%%%%%%%%%%%%%%%%%%%%%%%%%%%
\thispagestyle{empty}
\section*{Selbst\"andigkeitserkl\"arung}

Hiermit versichere ich, dass ich die vorliegende Masterarbeit 
selbst\"andig und nur mit den angegebenen Hilfsmitteln angefertigt habe und dass alle Stellen, die dem Wortlaut oder dem 
Sinne nach anderen Werken entnommen sind, durch Angaben von Quellen als 
Entlehnung kenntlich gemacht worden sind. 
Diese Masterarbeit wurde in gleicher oder \"ahnlicher Form in keinem anderen 
Studiengang als Pr\"ufungsleistung vorgelegt. 

\vskip 3cm
%Tübingen, 31.01.2023 \\
Ort, Datum	\hfill Unterschrift \hfill 
%%%%%%%%%%%%%%%%%%%%%%%%%%%%%%%%%%%%%%%%%%%%%%%%%%%%%%%%%%%%%%%%%%%%%%%%%%%%%
%%% Ende
%%%%%%%%%%%%%%%%%%%%%%%%%%%%%%%%%%%%%%%%%%%%%%%%%%%%%%%%%%%%%%%%%%%%%%%%%%%%%

\end{document}

